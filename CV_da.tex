\documentclass[12pt,a4paper,notitlepage]{article}
\usepackage[utf8]{inputenc}
\usepackage{amsmath}
\usepackage{amsfonts}
\usepackage{amssymb}
\usepackage{tabularx}

\usepackage{fancyhdr}
\usepackage{lastpage}
\pagestyle{fancy}
\fancyhf{}

\usepackage{geometry}
\newgeometry{margin=2cm}

\renewcommand{\headrulewidth}{0pt}
\renewcommand{\footrulewidth}{0pt}

\usepackage{titling}
\setlength{\droptitle}{-5em}
\cfoot{Side \thepage \hspace{1pt} af \pageref{LastPage}}
\usepackage[danish]{babel}
\author{Jens Egholm Pedersen, født 1988
\\ \texttt{jensegholm@protonmail.com}
}

\makeatletter
\let\ps@plain\ps@fancy
\makeatother

\title{Curriculum Vitae}
\begin{document}
\maketitle

\section*{Uddannelse}
\begin{tabularx}{\textwidth}{l X}
2016 - 2018 & Stud. MSc. IT \& Cognition, Københavns Universitet \\
2014 - 2014 & Kursusforløb i acceleratorfysik ved CERN (\texttt{AXEL-2015}) \\
2013 - 2013 & ERASMUS udvekslingsprogram 2013,
              Ècole d'Ingenieurs en Informatique, EPITA, Paris \\
2011 - 2014 & BSc. softwareudvikling, IT-Universitetet i København \\
2008 - 2011 & BSc. statskundskab, Aarhus Universitet \\
2004 - 2007 & Gymnasium, Esbjerg Gymnasium \& HF
\end{tabularx}

\section*{Ansættelser}
\begin{tabularx}{\textwidth}{l X}
2016 -      & \textbf{Adjunkt ved Copenhagen Business Academy} \\
            & Jeg har afviklet undervissforløb i tråd- og webprogrammering,
              funktionel programmering, relationelle databaser, machine learning
              samt design og drift af store systemer. Sammen med mine kolleger
              har vi udviklet nyt læringsmateriale i samtlige undervisningsforløb.
              Undervisningsen har kombineret traditionel teori med
              moderne praksis, og jeg har modtaget stor ros både
              for det faglige indhold og min personlige undervisningsstil.\\
            & En anbefaling kan sendes efter forespørgsel. \\
2016 -      & \textbf{CTO og medstifter, Mobilized Construction} \\
            & Som CTO har
              jeg designet og implementeret de digitale løsninger, der ligger til
              grund for firmaets videre vækst. Derudover har jeg det ledet hold af
              softwareudviklere fordelt i Kenya, Canada, USA og Wales. \\
2014 - 2016 & \textbf{Software ingeniør ved CERN} \\
            & Hos CERN udviklede og drev jeg overvågningsinfrastruktur for
              de kontrolsystemer, der driver CERNs acceleratorkompleks.
              Derudover var jeg involveret i udviklingen af testprotokoller for
              deres største accelerator (LHC), samt en prototype af et nyt
              real-time kontrolsystem i Java. \\
            & En anbefaling kan sendes efter forespørgsel. \\
2013 - 2013 & \textbf{Instruktor i kurset \textit{mobile og distribuerede systemer} ved ITU} \\
            & Her forelæste jeg i principper og forskning omkring mobile og
              distribuerede IT systemer, bidrog til at udvikle nyt
              undervisningsmateriale samt overså øvelsestimer. \\
2012 - 2014 & \textbf{Studenterprogrammør for forskningsgruppen DemTech (\texttt{demtech.dk})} \\
            & Her udviklede jeg et system bygget på 3G routere til at måle
              kølængder under kommunalvalget i 2013. Systemet blev rullet up på
              fem valgsteder. \\
2011 - 2013 & \textbf{Ambassadør for IT-Universitetet i Copenhagen} \\
            & Her underviste i basale programmeringsprincipper og afholdt
              workshops for kommende ITU studerende.
\end{tabularx}

\section*{Frivilligt arbejde}
\begin{tabularx}{\textwidth}{l X}
2014 - 2016 & \textbf{Guide og forelæser for besøgende til CERN} \\
            & Som guide blev jeg instrueret i forskellige dele af CERNs
              kontrolinfrastruktur samt de teoretiske principper bag. Jeg
              viste besøgende fra højere uddannelsesinstitutioner rundt i CERNs
              kontrolcenter, kryogene faciliteter samt præsenterede
              grundlæggende partikelfysik for dansker gymnasieklasser. \\
2011 - 2014 & \textbf{Aktiv hos den studenterpolitiske forening (StupIT) ved ITU} \\
            & Som medlem af StupIT var jeg med til at grundlægge en støttepulje
              til andre foreniger på ITU, samt grundlægge deres vedtægter. \\
2010 - 2011 & \textbf{Lektiehjælper hos Lektier-Online, Statsbibliotektet i Aarhus} \\
            & Som lektiehjælper hjalp jeg gymnasieelever over hele Danmark i
              matematik, dansk og samfundsfag. Sammen med ledelsen præsenterede
              jeg projektet for den daværende minister for integration og kirke. \\
2006 - 2007 & \textbf{Bestyrelsesmedlem ved Danske Gymnasieelevers Sammenslutning} \\
2005 - 2007 & \textbf{Elevrådsformand ved Esbjerg Gymnasium \& HF} \\
\end{tabularx}

\end{document}
