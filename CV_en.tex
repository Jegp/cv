
%------------------------------------
% Dario Taraborelli
% Typesetting your academic CV in LaTeX
%
% URL: http://nitens.org/taraborelli/cvtex
% DISCLAIMER: This template is provided for free and without any guarantee 
% that it will correctly compile on your system if you have a non-standard  
% configuration.
% Some rights reserved: http://creativecommons.org/licenses/by-sa/3.0/
%------------------------------------

%!TEX TS-program = xelatex
%!TEX encoding = UTF-8 Unicode

\documentclass[11pt, a4paper]{article}
\usepackage{fontspec} 

% DOCUMENT LAYOUT
\usepackage{geometry} 
\geometry{a4paper, textwidth=5.5in, textheight=8.5in, marginparsep=7pt, marginparwidth=.6in}
\setlength\parindent{0in}

% COMMENTS
\usepackage{verbatim}

% FONTS
\usepackage[usenames,dvipsnames]{xcolor}
\usepackage{xunicode}
\usepackage{xltxtra}
\defaultfontfeatures{Mapping=tex-text}
%\setromanfont [Ligatures={Common}, Numbers={OldStyle}, Variant=01]{Linux Libertine O}
%\setmonofont[Scale=0.8]{Monaco}
%%% modified by Karol Kozioł for ShareLaTeX use
\setmainfont[
  Ligatures={Common}, Numbers={OldStyle}, Variant=01,
  BoldFont=LinLibertine_RB.otf,
  ItalicFont=LinLibertine_RI.otf,
  BoldItalicFont=LinLibertine_RBI.otf
]{LinLibertine_R.otf}
\setmonofont[Scale=0.8]{DejaVuSansMono.ttf}

% ---- CUSTOM COMMANDS
\chardef\&="E050
\newcommand{\html}[1]{\href{#1}{\scriptsize\textsc{[html]}}}
\newcommand{\pdf}[1]{\href{#1}{\scriptsize\textsc{[pdf]}}}
\newcommand{\doi}[1]{\href{#1}{\scriptsize\textsc{[doi]}}}
% ---- MARGIN YEARS
\usepackage{marginnote}
\newcommand{\amper{}}{\chardef\amper="E0BD }
\newcommand{\years}[1]{\marginnote{\scriptsize #1}}
\renewcommand*{\raggedleftmarginnote}{}
\setlength{\marginparsep}{7pt}
\reversemarginpar

% HEADINGS
\usepackage{sectsty} 
\usepackage[normalem]{ulem} 
\sectionfont{\mdseries\upshape\Large}
\subsectionfont{\mdseries\scshape\normalsize} 
\subsubsectionfont{\mdseries\upshape\large} 

% PDF SETUP
% ---- FILL IN HERE THE DOC TITLE AND AUTHOR
\usepackage[%dvipdfm, 
bookmarks, colorlinks, breaklinks, 
% ---- FILL IN HERE THE TITLE AND AUTHOR
	pdftitle={Albert Einstein - vita},
	pdfauthor={My name},
	pdfproducer={http://nitens.org/taraborelli/cvtex}
]{hyperref}  
\hypersetup{linkcolor=blue,citecolor=blue,filecolor=black,urlcolor=MidnightBlue} 

\begin{document}

{\LARGE Jens Egholm Pedersen}\\[1cm]
  Unit for Cognitive Neuroscience\\
  University of Copenhagen\\[0.2cm]
  Øster Farimagsgade 2A\\
  1353 Copenhagen K, Denmark\\[.2cm]
Phone: \texttt{+45 25122752}\\
Email: \href{mailto:jensegholm@protonmail.com}{jensegholm@protonmail.com}\\

\vfill 
Born: 8. August 1988---Varde, Denmark\\
Nationality: Danish

\section*{Current position}
\emph{External lecturer}, IT-University of Copenhagen, Denmark

\section*{Education}
\noindent
\subsection*{University}
\years{2018}\textbf{MSc in Computer Science}, University of Copenhagen\\
\emph{Advisors: Martin Elsman and Jesper Mogensen}\\
\years{2015}\textbf{BSc in Computer Science}, IT-University of Copenhagen\\
\emph{Advisor: Peter Sestoft (IT-University of Copenhagen), Vito Baggiolini
(CERN)}\\
\years{2011}\textbf{BSc in Political Science}, University of Aarhus\\
\emph{Advisor: Tore V. Olsen}

\subsection*{Training}
\noindent
\years{2018}HBP School: ``The Brain Simulation Platform of the Human Brain
Project'' --- \emph{Palermo, Italy}\\
\years{2018}2\textsuperscript{nd} HBP Student Conference: ``Transdiscilinary
Research Linking Neuroscience, Brain Medicine and Computer Science'' ---
\emph{Ljubljana, Slovenia}\\
\years{2017}2\textsuperscript{nd} Young Researchers Event, Human Brain Project ---
\emph{Geneva, Switzerland}\\
\years{2014}AXEL course on particle accelerator physics, CERN

\subsection*{Languages}
\emph{Danish} (C2, native)\\
\emph{English} (C2, fully proficient)\\
\emph{French} (B1, working basic conversational) \\
\emph{German} (A2, basic)

\section*{Appointments held}
\noindent
\years{2018-}\textbf{External lecturer}, IT-University of Copenhagen, Denmark\\
At ITU I plan and teach a seminar on programming, programming tools and large-scale
visualisation techniques \\[0.5cm]
\years{2016-2018}\textbf{Chief Technology Officer}, Mobilized Construction, Kenya\\
At Mobilized Construction I built the infrastructure required for their
large-scale operations, but are now focusing on a managerial role to
the teams in Wales and Kenya. I have supervised several student projects
on machine learning and data science from the University of Southern
California.\\[0.5cm]
\years{2016-2018}\textbf{Assistant professor}, Copenhagen Business Academy,
Denmark\\
As an assistant professor I developed materials and lectured in
a range of topics, including functional programming,
machine learning and design and operation of large systems.
I received high praise for the professional content as well as
my communication and pedagogical skills.\\[0.5cm]
\years{2014-2016}\textbf{Software engineer}, CERN, Switzerland\\
At CERN I was driving the development and maintenance of
a monitoring toolchain for the accelerator control system complex.
My contract was extended to work on a testbed for the Large Hadron
Collider (LHC) accelerator, with an emphasis on real-time critical
systems architecture in Java. \\[0.5cm]
\years{2013}\textbf{Instructor}, Mobile and Distributed Systems, IT-University of
Copenhagen, Denmark\\
I lectured on principles and research on mobile and distributed
systems and participated in the development of new teaching
materials. \\[0.5cm]
\years{2012-2014}\textbf{Research assistant}, DemTech research group, Denmark\\
As a student programmer I developed a 3G based router system to
measure the length of election queues. Used for nation-wide
research and deployed at several polling places in the Danish national
election of 2013.

\section*{Honours \& awards}
\noindent
\years{2017}Finalist, Founders of Tomorrow
(\href{https://foundersoftomorrow.com}{FoundersOfTomorrow.com})\\
\years{2013}ERASMUS exchange programme, Ècole d'Ingenieurs en Informatique, Paris, France\\

\pagebreak

\section*{Publications and talks}
\noindent

\subsection*{Journal articles}
\years{2018}\textbf{Understanding the neurocognitive organization as strategies rather
than functions: Implications for neurological research}\\
Mogensen, J., Dauggaard, N., Kitsios, S., Pedersen, J. E., Overgaard, M.
\emph{EC Neurology}

\subsection*{Peer-reviewed conference papers}
\years{2015}\textbf{Introducing RepoCad - A prototype of the Internet of Digital
Design}\\
Jackson, O. E. and Pedersen, J. E.\\
\emph{eCAADe}, Volume 33

\subsection*{Peer-reviewed conference abstracts}
\years{2018}\textbf{Volr: A declarative interface language for neural
computation} --- \href{https://indico-jsc.fz-juelich.de/event/71/}{NEST
conference} abstract, poster\\
Pedersen, J. E. and Pehle, Christian,\\
\emph{NEST conference: A Forum for Users and Developers}

\subsection*{Theses and dissertations}
\years{2018}\textbf{Modelling neural learning systems --- Cognitive models in third
generation neural networks}\\
\emph{Master's thesis, University of Copenhagen}\\[0.4cm]
\years{2015}\textbf{Predictable firm real-time Java}\\
\emph{Bachelor's Dissertation, CERN and IT-University of Copenhagen}\\[0.4cm]
\years{2011}\textbf{Proxy voting in the European Union}\\
\emph{Bachelor's thesis, Aarhus University}

\subsection*{Conference presentations}
\years{2018}\textbf{Experimental neural systems modelling with Jupyter
Notebooks} ---
\emph{HBP School} \\
Pedersen, J. E. \\[0.4cm]
\years{2018}\textbf{Volr: A declarative interface language for neural
computation}
--- \emph{NEST conference}\\
Pedersen, J. E. and Pehle, C. 

\section*{Voluntary work}
\noindent
\years{2018}Lead developer and maintainer of Volr, a Haskell-based environment
for the construction, evaluation and analysis of neural networks
(\href{https://github.com/volr/}{github.com/volr}). \\
\years{2014-2016} Tour guide for visitors around the CERN complex, where
I was introduced to the different parts of the
control system in the accelerator complex, the cryogenics
facilities and particle physics. I held presentations
on all of the above for visitors from all around the globe. \\
\years{2008-2015}Co-creator and lead programmer of the collaborative drawing-tool
epoCad, now published as open-source
(\href{https://github.com/selftiesoftware}{github.com/selftiesoftware}).

%\section*{Passions}
%Cognitive science and philosophy of the mind. Has a personal goal to augment
%his cognitive capacities to better encompass the challenges (and solutions) of tomorrow.
%\vskip4pt
%\noindent
%Volunteer work. Believes in a strong social responsibility for the immediate community and society as a whole.
%\vskip4pt
%\noindent
%Device hacking. Rutinely changes the firmware and software of IT around him
%to support daily routines.
%\vskip4pt
%\noindent
%Music. Have been playing the violin since age 6 and piano since age 12.
%\vskip4pt
%%\noindent
%Hiking. Primarily in the Alps, Norway and Sweden.
%\vskip4pt
%\noindent

\end{document}
