%------------------------------------
% Dario Taraborelli
% Typesetting your academic CV in LaTeX
%
% URL: http://nitens.org/taraborelli/cvtex
% DISCLAIMER: This template is provided for free and without any guarantee 
% that it will correctly compile on your system if you have a non-standard  
% configuration.
% Some rights reserved: http://creativecommons.org/licenses/by-sa/3.0/
%------------------------------------

%!TEX TS-program = xelatex
%!TEX encoding = UTF-8 Unicode

\documentclass[11pt, a4paper]{article}
\usepackage{fontspec} 

% DOCUMENT LAYOUT
\usepackage{geometry} 
\geometry{a4paper, textwidth=5.8in, textheight=10.2in, marginparsep=7pt, marginparwidth=.6in}
\setlength\parindent{0in}

% COMMENTS
\usepackage{verbatim}

% FONTS
\usepackage[usenames,dvipsnames]{xcolor}
\usepackage{xunicode}
\usepackage{xltxtra}
\defaultfontfeatures{Mapping=tex-text}
%\setromanfont [Ligatures={Common}, Numbers={OldStyle}, Variant=01]{Linux Libertine O}
%\setmonofont[Scale=0.8]{Monaco}
%%% modified by Karol Kozioł for ShareLaTeX use
\setmainfont[
  Ligatures={Common}, Numbers={OldStyle}, Variant=01,
  BoldFont=LinLibertine_RB.otf,
  ItalicFont=LinLibertine_RI.otf,
  BoldItalicFont=LinLibertine_RBI.otf
]{LinLibertine_R.otf}
\setmonofont[Scale=0.8]{DejaVuSansMono.ttf}

% ---- CUSTOM COMMANDS
\chardef\&="E050
\newcommand{\html}[1]{\href{#1}{\scriptsize\textsc{[html]}}}
\newcommand{\pdf}[1]{\href{#1}{\scriptsize\textsc{[pdf]}}}
\newcommand{\doi}[1]{\href{#1}{\scriptsize\textsc{[doi]}}}
% ---- MARGIN YEARSWho?	Uni	City	State
\usepackage{marginnote}
\newcommand{\amper{}}{\chardef\amper="E0BD }
\newcommand{\years}[1]{\marginnote{\scriptsize #1}}
\renewcommand*{\raggedleftmarginnote}{}
\setlength{\marginparsep}{7pt}
\reversemarginpar

% HEADINGS
\usepackage{sectsty} 
\usepackage[normalem]{ulem} 
\sectionfont{\mdseries\upshape\Large}
\subsectionfont{\mdseries\scshape\normalsize} 
\subsubsectionfont{\mdseries\upshape\large} 

% PDF SETUP
\usepackage[%dvipdfm, 
bookmarks, colorlinks, breaklinks, 
	pdftitle={Jens Egholm Pedersen - curriculum vitae},
	pdfauthor={Jens Egholm Pedersen},
	pdfproducer={https://github.com/jegp}
]{hyperref}  
\hypersetup{linkcolor=blue,citecolor=blue,filecolor=black,urlcolor=MidnightBlue} 

\begin{document}

{\LARGE Jens Egholm Pedersen}\\[.6cm]
\begin{tabular}{@{}l l l}
      Contact: & \href{mailto:jeped@kth.se}{jeped@kth.se}                  \\
      Website: & \href{https://jepedersen.dk}{jepedersen.dk}\hspace{1.6cm}
      Mastodon: \href{https://mastodon.social/@jegp}{@jegp}\hspace{1.6cm}
      GitHub: \href{https://github.com/jegp/}{@jegp}
\end{tabular}


\section*{Education}
\years{2019-2024 (est.)}\textbf{PhD in Computer Science}, KTH Royal Institute of Technology, Sweden \\
Thesis: \textbf{Neuromorphic computing in space and time}\\
Advisors:
\href{https://www.kth.se/profile/conr}{Jörg Conradt} (KTH) and
\href{https://www.kth.se/profile/arvindku}{Arvind Kumar} (KTH, Karolinska Institute) \\
Courses: Machine learning, Computational Neuroscience, Computer vision, Mathematics, Philosophy of Science, Supervision, Communication, and Teaching \\
\years{2016-2019}\textbf{MSc in IT} \& \textbf{Cognition}, University of Copenhagen\\
Thesis: Modelling learning systems in spiking and artificial neural networks.\
\\
\years{2011-2015}\textbf{BSc in Computer Science}, IT-University of Copenhagen\\
Thesis: Predictable firm real-time Java. In collaboration with CERN.
\\
\years{2009-2011}\textbf{BSc in Political Science}, University of Aarhus, Denmark\\
Thesis: Proxy voting in the European Union.

\section*{Peer-reviewed publications}
\noindent

\years{}\textbf{J. E. Pedersen}*, S. Abreau *, M. Jobst, G. Lenz, F. Vittorio, F. C. Bauer, P. Zhou, D. Muir, B. Vogginger, K. Heckel,
T. Stewart, S. Shankar, J. Eshragian, S. Sheik. (* shared authorship) ``The Neuromorphic Intermediate Representation''. \emph{Under review} \\
\years{}\textbf{J. E. Pedersen}, J. Conradt, T. Lindeberg. ``Spatio-temporal covariance properties for neuromorphic computing''. \emph{Under review} \\
\years{2023} \textbf{J. E. Pedersen}, R. Singhal \& J. Conradt.
``Translation and Scale Invariance for Event-Based Object tracking.''
\href{https://dl.acm.org/conference/nice}{NICE - Neuro Inspired Computational Elements Conference} \\
\years{2023} \textbf{J. E. Pedersen} \& J. Conradt.
``AEStream: Accelerated event-based processing with coroutines.''
\href{https://dl.acm.org/conference/nice}{NICE - Neuro Inspired Computational Elements Conference} \\
\years{2023} J. P. Romero B., L. A. Plana, A. Rowley, M. Hessel, \textbf{J. E. Pedersen}, S. Furber, J. Conradt.
``A High-Throughput Low-Latency Interface Board for SpiNNaker-in-the-loop Real-Time Systems.''
\href{https://icons.ornl.gov/}{ICONS - International Conference on Neuromorphic Systems} \\
\years{2022} J. Covelo, S. Diaz, A. Geminiani, T. Kirchner, C. Lupascu,
P. Ochang, T. Özden, \textbf{J. E. Pedersen}, I. Reiten, A. Yegenoglu, M. Gran, J. Kathrein, Tabea Kirchner, P. Ochang, F. Vogel.
``Proceedings for the 6th HBP Student Conference on Interdisciplinary Brain Research.''
\emph{Frontiers Event Abstracts} \\
\years{2018} J. Mogensen, N. Dauggaard, S. Kitsios, \textbf{J. E. Pedersen}, M. Overgaard.
``Understanding the neurocognitive organization as strategies rather than functions: Implications for neurological research.''
\emph{EC Neurology}

\section*{Open-access publications}
\years{2022} \textbf{J. E. Pedersen}, J. P. Romero B. \& J. Conradt,
``Coordinate regression with Spiking Neural Networks.''
Workshop on \href{https://neal2022.tetzlab.com/}{Neuromorphic Algorithms}, \\
\years{2020} \textbf{J. E. Pedersen}*, C. Pehle*, (* shared authorship)
``Norse - Spiking neural network for deep learning.''
\href{https://zenodo.org/record/4422025}{Zenodo}
\\
\years{2018} \textbf{J. E. Pedersen} \& C. Pehle,
``Volr: A declarative interface language for neural computation.''
\href{https://indico-jsc.fz-juelich.de/event/71/}{NEST conference: A Forum for Users and Developers}\\

\section*{Honours, Grants \& awards}
\years{2024}NSF AccelNet NeuroPAC Fellowship with Professor Sadasivan Shankar, Stanford University.\\
\years{2022 - 2024}Compute grant for Swedish National Infrastructure for Computing \\
\years{2018}Winner, \textbf{Edge of Government Award} for the startup Mobilized
Construction, Prime Minister of the United Arab Emirates.\\
\years{2017}Finalist, \textbf{Founders of Tomorrow} startup programme
(\href{https://foundersoftomorrow.com}{FoundersOfTomorrow.com}).\\

\section*{Select appointments held}
\begin{comment}
\noindent
\years{2019-}\textbf{Doctoral student}, KTH Royal Institute of Technology,
Stockholm, Sweden.\\
Following the title of my thesis ``Neuromorphic control systems'', I work with
sparse, event-driven, and massively parallel brain-inspired algorithms to
understand and construct goal-oriented and autonomous control systems.\\
\textit{Advisors}: \href{https://www.kth.se/profile/conr}{Jörg Conradt} and
\href{https://www.kth.se/profile/arvindku}{Arvind Kumar}.
\\
\end{comment}
\years{2018-2021}\textbf{External lecturer}, IT-University of Copenhagen, Denmark\\
I planned and taught courses on Python and data science with outstanding reviews. \\
\years{2016-2019}\textbf{Adjunct professor}, Copenhagen Business Academy,
Denmark\\
I lectured on machine learning, artificial intelligence, business analytics and distributed systems infrastructure with outstanding reviews.\\
\years{2016-2019}\textbf{Chief Technology Officer}, Mobilized Construction, Denmark, Kenya\\
I designed a globally distributed software stack, managed teams in the US, Wales, Kazaksthan, and Kenya, and supervised projects from several universities.\\
\years{2014-2016}\textbf{Software engineer}, CERN, Switzerland\\
I developed and maintained a monitoring toolchain and testbed for the Large
Hadron Collider.

\section*{Talks}
\years{2024}NIR: A unified instruction set for brain-inspired computing --- \emph{SNUFA - Spiking Neural networks as Universal Function Approximators \& separately at an Open Neuromorphic Workshop} \\
\years{2023}Translation and Scale Invariance for Event-Based Object tracking ---
\emph{NICE - Neuro Inspired Computational Elements Conference} \\
\years{2023}The need for neuromorphic abstractions ---
\emph{Open Neuromorphic event on Neuromorphic open-source computing} \\
\years{2023}AEStream: Accelerated event-based processing with coroutines ---
\emph{NICE - Neuro Inspired Computational Elements Conference} \\
\years{2022 \& 2023}Optimizing spiking neural networks with Norse ---
\emph{6th \& 7th HBP Student Conference} \\
\years{2021}Norse: A library for gradient-based learning in Spiking Neural Networks --- Workshop on \emph{SNUFA - Spiking Neural networks as Universal Function Approximators} \\

\vspace*{-5mm}
\section*{Selected student projects}
\years{2023}Oskar Strömberg, \emph{Event-based vision with spiking vision transformers}, KTH, MSc. \\
\years{2022}Philpp Mondorff, \emph{Spiking Reinforcement Learning for Robust Robot Control}, KTH, MSc.\\
\years{2022}Merlin Sewina, \emph{Decoding EEG with Spiking and Artificial Neural Networks}, KTH, MSc.\\
\years{2020}Mikkel Ziemer, \emph{Building CERN's Control System}, CPHBusiness, BSc.\\

\section*{Contributions to the community}
\noindent
\years{2024}\textbf{Conference committee member}---\emph{NICE - Neuro Inspired Computational Elements Conference} \\
\years{2022}\textbf{Real-time neuromorphics with Norse and AEStream}---\emph{\href{https://jepedersen.dk/slides/202209_neurotech/index.html}{Neurotech Tutorial}} \\
\years{2020-2023}\textbf{Conference committee member}---5th, 6th, and 7th Human Brain Project Student Conference on
on Interdisciplinary Brain Research.\\
% \years{2020-2023}\textbf{Student ambassador}---Human Brain Project.\\
\years{2020-}\textbf{AEStream}---a library for the efficient streaming of event-based data representations (\href{https://github.com/aestream}{github.com/aestream}) \\
\years{2018-}\textbf{Norse}---a PyTorch-based deep learning library for spiking neural networks (\href{https://github.com/norse/}{github.com/norse}). \\

\begin{comment}
\section*{Technical training}

\years{2018}``The Brain Simulation Platform of the Human Brain
Project'', Human Brain Project summer school  --- \emph{Palermo, Italy}\\
\years{2018}``Transdisciplinary
Research Linking Neuroscience, Brain Medicine and Computer Science'', Human Brain Project
young researchers event ---
\emph{Ljubljana, Slovenia}\\
\years{2017}2\textsuperscript{nd} Young Researchers Event, Human Brain Project ---
\emph{Geneva, Switzerland}\\
\years{2014}AXEL course on particle accelerator physics --- \emph{CERN, Switzerland}

\section*{Voluntary work}
\years{2014-2016} Lecturer and tour guide for visitors in and around the CERN complex.
I was given in-depth introductions to the different parts of the
control system in the accelerator complex, the cryogenics
facilities and particle physics. \\
\years{2008-2015}Co-creator and lead programmer of the collaborative drawing-tool RepoCad, now published as open-source
(\href{https://github.com/selftiesoftware}{github.com/selftiesoftware}).

\section*{Languages}
\emph{Danish} (C2, native)\\
\emph{English} (C2, fully proficient)\\
\emph{French} (B1, working basic conversational) \\
\emph{German} (B1, working basic conversational)

\pagebreak
\section*{References}
The following professionals kindly offered their availability for personal
references and questions:
\begin{enumerate}
      \item Jesper Mogensen, Professor, Unit for Cognitive Neuroscience, Copenhagen University
            \\ Email: \texttt{jesper.mogensen@psy.ku.dk}
            \\ Phone: \texttt{(+45) 3532 4873}
      \item Lars Bogetoft, Head of IT Program, Copenhagen Business Academy (former)
            \\ Email: \texttt{lars.bogetoft@gmail.com}
            \\ Phone: \texttt{(+45) 5185 0497}
      \item Kevin Lee, CEO, Mobilized Construction
            \\ Email: \texttt{kevin@mobilizedconstruction.com}
            \\ Phone: \texttt{(+44) 794 046 3174}
\end{enumerate}
\end{comment}

\end{document}
