%------------------------------------
% Dario Taraborelli
% Typesetting your academic CV in LaTeX
%
% URL: http://nitens.org/taraborelli/cvtex
% DISCLAIMER: This template is provided for free and without any guarantee 
% that it will correctly compile on your system if you have a non-standard  
% configuration.
% Some rights reserved: http://creativecommons.org/licenses/by-sa/3.0/
%------------------------------------

%!TEX TS-program = xelatex
%!TEX encoding = UTF-8 Unicode

\documentclass[11pt, a4paper]{article}
\usepackage{fontspec} 

% DOCUMENT LAYOUT
\usepackage{geometry} 
\geometry{a4paper, textwidth=5.5in, textheight=10in, marginparsep=7pt, marginparwidth=.6in}
\setlength\parindent{0in}

% COMMENTS
\usepackage{verbatim}

% FONTS
\usepackage[usenames,dvipsnames]{xcolor}
\usepackage{xunicode}
\usepackage{xltxtra}
\defaultfontfeatures{Mapping=tex-text}
%\setromanfont [Ligatures={Common}, Numbers={OldStyle}, Variant=01]{Linux Libertine O}
%\setmonofont[Scale=0.8]{Monaco}
%%% modified by Karol Kozioł for ShareLaTeX use
\setmainfont[
  Ligatures={Common}, Numbers={OldStyle}, Variant=01,
  BoldFont=LinLibertine_RB.otf,
  ItalicFont=LinLibertine_RI.otf,
  BoldItalicFont=LinLibertine_RBI.otf
]{LinLibertine_R.otf}
\setmonofont[Scale=0.8]{DejaVuSansMono.ttf}

% ---- CUSTOM COMMANDS
\chardef\&="E050
\newcommand{\html}[1]{\href{#1}{\scriptsize\textsc{[html]}}}
\newcommand{\pdf}[1]{\href{#1}{\scriptsize\textsc{[pdf]}}}
\newcommand{\doi}[1]{\href{#1}{\scriptsize\textsc{[doi]}}}
% ---- MARGIN YEARS
\usepackage{marginnote}
\newcommand{\amper{}}{\chardef\amper="E0BD }
\newcommand{\years}[1]{\marginnote{\scriptsize #1}}
\renewcommand*{\raggedleftmarginnote}{}
\setlength{\marginparsep}{7pt}
\reversemarginpar

% HEADINGS
\usepackage{sectsty} 
\usepackage[normalem]{ulem} 
\sectionfont{\mdseries\upshape\Large}
\subsectionfont{\mdseries\scshape\normalsize} 
\subsubsectionfont{\mdseries\upshape\large} 

% PDF SETUP
\usepackage[%dvipdfm, 
bookmarks, colorlinks, breaklinks, 
	pdftitle={Jens Egholm Pedersen - curriculum vitae},
	pdfauthor={Jens Egholm Pedersen},
	pdfproducer={https://github.com/jegp}
]{hyperref}  
\hypersetup{linkcolor=blue,citecolor=blue,filecolor=black,urlcolor=MidnightBlue} 

\begin{document}

{\LARGE Jens Egholm Pedersen}\\[.6cm]
\begin{tabular}{@{}l l l}
   Contact: & \href{mailto:jens@jepedersen.dk}{jens@jepedersen.dk} \\
   Website: & \href{https://jepedersen.dk}{jepedersen.dk}\hspace{1.6cm}
   Twitter: \href{https://twitter.com/jensegholm}{@jensegholm}\hspace{1.6cm}
   GitHub: \href{https://github.com/jegp/}{@jegp}
\end{tabular}

\section*{Current position}
\years{2019-}\textbf{Doctoral student}, Department of Computer Science, 
Royal Institute of Technology, KTH, Sweden\\
Thesis: \textbf{Neuromorphic control systems}\\
\begin{tabular}{@{}r @{\hspace{0.1cm}} l}
   \emph{Advisors:} & \href{https://www.kth.se/profile/conr}{Jörg Conradt} (Department of Computer Science, KTH) and \\
      & \href{https://www.kth.se/profile/arvindku}{Arvind Kumar} (Department of Computer Science, KTH, Karolinska Institute)
\end{tabular}

\section*{Selected appointments held}
\begin{comment}
\noindent
\years{2019-}\textbf{Doctoral student}, KTH Royal Institute of Technology,
Stockholm, Sweden.\\
Following the title of my thesis ``Neuromorphic control systems'', I work with
sparse, event-driven, and massively parallel brain-inspired algorithms to 
understand and construct goal-oriented and autonomous control systems.\\
\textit{Advisors}: \href{https://www.kth.se/profile/conr}{Jörg Conradt} and 
\href{https://www.kth.se/profile/arvindku}{Arvind Kumar}.
\\[0.2cm]
\end{comment}
\years{2018-2020}\textbf{External lecturer}, IT-University of Copenhagen, Denmark\\
I planned and taught computer science courses on undergraduate and postgraduate level
with outstanding reviews. \\[0.2cm]
\years{2016-2019}\textbf{Adjunct professor}, Copenhagen Business Academy,
Denmark\\
I lectured on a range of topics, including machine learning, artificial
intelligence, business analytics and distributed systems infrastructure.
I received high praise for my pedagogical skills and ability to translate
complex academic subjects to students.\\[0.2cm]
\years{2016-2019}\textbf{Chief Technology Officer}, Mobilized Construction, Denmark, Kenya\\
I designed and implemented the software stack for globally distributed
software services and assisted in the development of Arduino-based sensor
systems.
I managed development teams in the US, Wales, Kazaksthan, and Kenya, and
personally supervised student projects from the University of Southern
California and the University of Waterloo.\\[0.2cm]
\years{2014-2016}\textbf{Software engineer}, CERN, Switzerland\\
I developed and maintained a monitoring toolchain and testbed for the Large
Hadron Collider (LHC).
I completed my bachelor's thesis about real-time critical systems architecture
in Java, jointly supervised between ITU and CERN.

\section*{Journal and conference publications}
\noindent

\years{2018}\textbf{Understanding the neurocognitive organization as strategies rather
than functions: Implications for neurological research}\\
Mogensen, J., Dauggaard, N., Kitsios, S., Pedersen, J. E., Overgaard, M.
\emph{EC Neurology}
\\[0.2cm]
\years{2015}\textbf{Introducing RepoCad - A prototype of the Internet of Digital
Design}\\
Jackson, O. E. \& Pedersen, J. E.\\
\emph{eCAADe}, Volume 33

\begin{comment}
\subsection*{Unpublished articles}
\years{2021}\textbf{A Language Approach to Metaplasticity and Learning in Neural Networks} \\
Pehle, C. \& Pedersen, J. E.\\
\emph{Unpublished, expected October 2021}, Computer Science, Mathematics
\end{comment}

\section*{Workshop papers and abstracts}
\years{2019}\textbf{Volr - A declarative approach to learning in spiking neural networks} 
--- \href{https://education.humanbrainproject.eu/web/3rd-hbp-student-conference}{NEST
conference}, abstract, poster\\
Pedersen, J. E. \& Pehle, C.,\\
\emph{3rd HBP Student Conference On Interdisciplinary Brain Research}
\\[0.2cm]
\years{2018}\textbf{Volr: A declarative interface language for neural computation}
--- \href{https://indico-jsc.fz-juelich.de/event/71/}{NEST
conference}, abstract, poster\\
Pedersen, J. E. \& Pehle, C.,\\
\emph{NEST conference: A Forum for Users and Developers}

\section*{Invited talks}
\years{2021}\textbf{Deep learning with spiking neural networks in Norse} ---
\emph{5th HBP Student Conference} \\[0.2cm]
\years{2018}\textbf{Experimental neural systems modelling with Jupyter
Notebooks} ---
\emph{HBP School}
\\[0.2cm]
\years{2018}\textbf{Volr: A declarative interface language for neural
computation}
--- \emph{NEST conference}

\section*{Education}
\years{2016-2019}\textbf{MSc in IT and Cognition}, University of Copenhagen, Denmark\\
Thesis: \textbf{Modelling learning systems in spiking and artificial neural networks}\\
\begin{tabular}{@{}r @{\hspace{0.1cm}} l}
\emph{Advisors:} & \emph{Martin Elsman (Department of Computer Science) and}\\
   & \emph{Jesper Mogensen (Unit for Cognitive Neuroscience, co-supervisor)}
\end{tabular}
\\[.2cm]
\years{2011-2015}\textbf{BSc in Computer Science}, IT-University of Copenhagen, Denmark\\
Thesis: \textbf{Predictable firm real-time Java}\\
\begin{tabular}{@{}r @{\hspace{0.1cm}} l}
\emph{Advisors:} & \emph{Peter Sestoft (IT-University of Copenhagen) and}\\
   & \emph{Vito Baggiolini (CERN)}
\end{tabular}
\\[.2cm]
\years{2009-2011}\textbf{BSc in Political Science}, University of Aarhus, Denmark\\
Thesis: \textbf{Proxy voting in the European Union}\\
\emph{Advisor: Tore V. Olsen (Department of Political Science)}

\section*{Honours \& awards}
\years{2018}Winner, \textbf{Edge of Government Award} for the startup Mobilized
Construction, Prime Minister of the United Arab Emirates.\\
\years{2017}Finalist, \textbf{Founders of Tomorrow} startup programme
(\href{https://foundersoftomorrow.com}{FoundersOfTomorrow.com}).\\
\years{2013}\textbf{ERASMUS exchange programme}, Ècole Pour l'Informatique et les
Techniques Avancées, Paris, France.

\section*{Selected student projects}
\years{2021}Merlin Sewina, \emph{Decoding EEG signals with Spiking and Artificial Neural Networks}, MSc.\\
\years{2020}Mikkel Ziemer, \emph{Building CERN's Control System}, BSc.\\
\years{2019}Kristjan R. Gásadal, \emph{Optimizing a Cross-platform Mobile Application
using Hashing to detect changes}, BSc.

\section*{Professional activities}
\noindent
\years{2020-}Conference committee member, 5th Human Brain Project Student Conference on
   on Interdisciplinary Brain Research.\\
\years{2020-}Student ambassador, Human Brain Project.\\
\years{2018-}Maintainer of Norse, a PyTorch-based
deep learning library for spiking neural networks
(\href{https://github.com/norse/}{github.com/norse}). \\

\begin{comment}
\section*{Technical training}

\years{2018}``The Brain Simulation Platform of the Human Brain
Project'', Human Brain Project summer school  --- \emph{Palermo, Italy}\\
\years{2018}``Transdisciplinary
Research Linking Neuroscience, Brain Medicine and Computer Science'', Human Brain Project
young researchers event ---
\emph{Ljubljana, Slovenia}\\
\years{2017}2\textsuperscript{nd} Young Researchers Event, Human Brain Project ---
\emph{Geneva, Switzerland}\\
\years{2014}AXEL course on particle accelerator physics --- \emph{CERN, Switzerland}

\section*{Voluntary work}
\years{2014-2016} Lecturer and tour guide for visitors in and around the CERN complex.
I was given in-depth introductions to the different parts of the
control system in the accelerator complex, the cryogenics
facilities and particle physics. \\
\years{2008-2015}Co-creator and lead programmer of the collaborative drawing-tool RepoCad, now published as open-source
(\href{https://github.com/selftiesoftware}{github.com/selftiesoftware}).

\section*{Languages}
\emph{Danish} (C2, native)\\
\emph{English} (C2, fully proficient)\\
\emph{French} (B1, working basic conversational) \\
\emph{German} (B1, working basic conversational)

\pagebreak
\section*{References}
The following professionals kindly offered their availability for personal
references and questions:
\begin{enumerate}
\item Jesper Mogensen, Professor, Unit for Cognitive Neuroscience, Copenhagen University 
\\ Email: \texttt{jesper.mogensen@psy.ku.dk}
\\ Phone: \texttt{(+45) 3532 4873}
\item Lars Bogetoft, Head of IT Program, Copenhagen Business Academy (former)
\\ Email: \texttt{lars.bogetoft@gmail.com}
\\ Phone: \texttt{(+45) 5185 0497}
\item Kevin Lee, CEO, Mobilized Construction
\\ Email: \texttt{kevin@mobilizedconstruction.com}
\\ Phone: \texttt{(+44) 794 046 3174}
\end{enumerate}
\end{comment}

\end{document}
