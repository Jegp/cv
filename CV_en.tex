%------------------------------------
% Dario Taraborelli
% Typesetting your academic CV in LaTeX
%
% URL: http://nitens.org/taraborelli/cvtex
% DISCLAIMER: This template is provided for free and without any guarantee 
% that it will correctly compile on your system if you have a non-standard  
% configuration.
% Some rights reserved: http://creativecommons.org/licenses/by-sa/3.0/
%------------------------------------

%!TEX TS-program = xelatex
%!TEX encoding = UTF-8 Unicode

\documentclass[11pt, a4paper]{article}
\usepackage{fontspec} 

% DOCUMENT LAYOUT
\usepackage{geometry} 
\geometry{a4paper, textwidth=5.8in, textheight=10.2in, marginparsep=7pt, marginparwidth=.6in}
\setlength\parindent{0in}

% COMMENTS
\usepackage{verbatim}

% FONTS
\usepackage[usenames,dvipsnames]{xcolor}
\usepackage{xunicode}
\usepackage{xltxtra}
\defaultfontfeatures{Mapping=tex-text}
%\setromanfont [Ligatures={Common}, Numbers={OldStyle}, Variant=01]{Linux Libertine O}
%\setmonofont[Scale=0.8]{Monaco}
%%% modified by Karol Kozioł for ShareLaTeX use
\setmainfont[
  Ligatures={Common}, Numbers={OldStyle}, Variant=01,
  BoldFont=LinLibertine_RB.otf,
  ItalicFont=LinLibertine_RI.otf,
  BoldItalicFont=LinLibertine_RBI.otf
]{LinLibertine_R.otf}
\setmonofont[Scale=0.8]{DejaVuSansMono.ttf}

% ---- CUSTOM COMMANDS
\chardef\&="E050
\newcommand{\html}[1]{\href{#1}{\scriptsize\textsc{[html]}}}
\newcommand{\pdf}[1]{\href{#1}{\scriptsize\textsc{[pdf]}}}
\newcommand{\doi}[1]{\href{#1}{\scriptsize\textsc{[doi]}}}
% ---- MARGIN YEARSWho?	Uni	City	State
\usepackage{marginnote}
\newcommand{\amper{}}{\chardef\amper="E0BD }
\newcommand{\years}[1]{\marginnote{\scriptsize #1}}
\renewcommand*{\raggedleftmarginnote}{}
\setlength{\marginparsep}{7pt}
\reversemarginpar

\usepackage[symbol]{footmisc}

% HEADINGS
\usepackage{sectsty} 
\usepackage[normalem]{ulem} 
\sectionfont{\mdseries\upshape\Large}
\subsectionfont{\mdseries\scshape\normalsize} 
\subsubsectionfont{\mdseries\upshape\large} 

\usepackage{titlesec}
\titlespacing*{\section} {0pt}{3ex plus 1ex minus .2ex}{1.3ex plus .2ex}
\titlespacing*{\subsection} {0pt}{2.25ex plus 1ex minus .2ex}{0.5ex plus .2ex}

% PDF SETUP
\usepackage[%dvipdfm, 
bookmarks, colorlinks, breaklinks, 
	pdftitle={Jens Egholm Pedersen - curriculum vitae},
	pdfauthor={Jens Egholm Pedersen},
	pdfproducer={https://jepedersen.dk}
]{hyperref}  
\hypersetup{linkcolor=blue,citecolor=blue,filecolor=black,urlcolor=MidnightBlue} 
\begin{document}

{\LARGE Jens Egholm Pedersen} \hspace{1cm} {\large Born: 1988, Denmark}\\[.4cm]
Contact: \href{mailto:jeped@kth.se}{jeped@kth.se}\hspace{1.2cm}
Website: \href{https://jepedersen.dk}{jepedersen.dk}\hspace{1.2cm}
GitHub: \href{https://github.com/jegp/}{@jegp}\hspace{1.2cm}


\section*{Education}
\years{2019-2025 (est.)}\textbf{PhD in Computer Science}, KTH Royal Institute of Technology, Sweden. \\
Thesis: \textbf{Neuromorphic computing in space and time}.\\
Advisors:
\href{https://www.kth.se/profile/conr}{Jörg Conradt} (KTH) and
\href{https://www.kth.se/profile/arvindku}{Arvind Kumar} (KTH, Karolinska Institute). \\
Research stay: Stanford University, supervised by \href{https://profiles.stanford.edu/sadasivan-shankar}{Sadasivan Shankar}.\\
\years{2016-2019}\textbf{MSc in IT} \& \textbf{Cognition}, University of Copenhagen.\\
Thesis: Modelling learning systems in spiking and artificial neural networks.\
\\
\years{2011-2015}\textbf{BSc in Computer Science}, IT-University of Copenhagen.\\
Thesis: Predictable firm real-time Java. In collaboration with CERN.
\\
\years{2009-2011}\textbf{BSc in Political Science}, University of Aarhus, Denmark.\\
Thesis: Proxy voting in the European Union.

\section*{Peer-reviewed publications {\small (* indicates equal authorship)}}
\years{2025}\textbf{J. E. Pedersen}, J. Conradt, \& T. Lindeberg. ``Covariant spatio-temporal receptive fields for neuromorphic computing''. \emph{In review}. \\
\years{2025}J. P. Romero B., D. Korakovounis, \textbf{J. E. Pedersen}, J. Conradt. ``Closed-loop neuromorphic air hockey player with millisecond reaction time''. \emph{In review}. \\
\years{2024}\textbf{J. E. Pedersen}*, S. Abreau*, J. Eshraghian, et al. ``The Neuromorphic Intermediate Representation''.
\href{https://www.nature.com/ncomms/}{Nature Communications} [IF: 17.69]. \\
\years{2024} S. Abreau*, \textbf{J. E. Pedersen}*, K. Heckel*, \& A. Pierro. ``Q-S5: Towards Quantized State Space Models.''
\href{https://icml.cc/virtual/2024/workshop/29962}{ICML - Next Generation of Sequence Modeling Architectures} [Acceptance rate: 30.5\%]. \\
\years{2024} S. Abreau, \textbf{J. E. Pedersen}.
``Neuromorphic Programming: Emerging Directions for Brain-Inspired Hardware''.
\href{https://iconsneuromorphic.cc/}{International Conference on Neuromorphic Computing Systems}. \\
\years{2024} A. Geminiani, J. Kathrein, ..., \textbf{J. E. Pedersen} et al. ``Multidisciplinary and collaborative training in neuroscience: Insights from the Human Brain Project Education Programme''. \href{https://link.springer.com/journal/12021}{Neuroinformatics} [IF: 2.7]. \\
\years{2023} \textbf{J. E. Pedersen}, R. Singhal, \& J. Conradt.
``Translation and Scale Invariance for Event-Based Object tracking''.
\href{https://niceworkshop.org/}{Neuro Inspired Computational Elements Conference (NICE)}. \\
\years{2023} \textbf{J. E. Pedersen} \& J. Conradt.
``AEStream: Accelerated event-based processing with coroutines.''
\href{https://niceworkshop.org/}{Neuro Inspired Computational Elements Conference (NICE)}. \\
\years{2023} J. P. Romero B., L. A. Plana, A. Rowley, M. Hessel, \textbf{J. E. Pedersen}, S. Furber, J. Conradt.
``A High-Throughput Low-Latency Interface Board for SpiNNaker-in-the-loop Real-Time Systems.''
\href{https://icons.ornl.gov/}{ICONS - International Conference on Neuromorphic Systems}. \\
\years{2018} J. Mogensen, N. Dauggaard, S. Kitsios, \textbf{J. E. Pedersen}, M. Overgaard.
``Understanding the neurocognitive organization as strategies rather than functions: Implications for neurological research.''
\emph{EC Neurology}.

\section*{Open-access publications {\small (* indicates equal authorship)}}
\years{2024} \textbf{J. E. Pedersen}, R. Singhal, \& J. Conradt, ``Event dataset generation for Galilean and affine transformations''. \href{https://zenodo.org/records/11063678}{Zenodo}. \\
\years{2022} \textbf{J. E. Pedersen}, J. P. Romero B. \& J. Conradt,
``Coordinate regression with Spiking Neural Networks.''
Workshop on \href{https://neal2022.tetzlab.com/}{Neuromorphic Algorithms}. \\
\years{2022} J. Turner, \textbf{J. E. Pedersen}, J. Conradt, \& T Nowotny, ``Event-based dataset for classification and pose estimation.''
\href{https://niceworkshop.org/}{Neuro Inspired Computational Elements Conference (NICE)}. \\
\years{2020} \textbf{J. E. Pedersen}*, C. Pehle*, ``Norse - Spiking neural network for deep learning.''
\href{https://zenodo.org/record/4422025}{Zenodo}.

\section*{Honours, Grants \& awards}
\years{2025}\href{https://www.mahowaldprize.org/prize-awards/prizes-2025}{Mahowald Early Career Award} for work on the Neuromorphic Intermediate Representation.\\
\years{2024}\href{https://www.neuropac.info/fellowships/}{NSF AccelNet NeuroPAC Fellowship} with Professor Sadasivan Shankar, Stanford University.\\
\years{2022 - 2025}Compute grant for Swedish National Infrastructure

\section*{Select appointments held}
\years{2018-2025}\textbf{External lecturer}, IT-University of Copenhagen, Denmark\\
I planned and taught courses on Python and data science with outstanding reviews. \\
\years{2016-2019}\textbf{Adjunct professor}, Copenhagen Business Academy,
Denmark\\
I lectured on machine learning, artificial intelligence, business analytics and distributed systems infrastructure with outstanding reviews.\\
\years{2016-2019}\textbf{Chief Technology Officer}, Mobilized Construction, Denmark, Kenya\\
I designed a globally distributed software stack, managed teams in the US, Wales, Kazaksthan, and Kenya, and supervised projects from several universities.\\
\years{2014-2016}\textbf{Software engineer}, CERN, Switzerland\\
I developed and maintained a monitoring toolchain and testbed for the Large
Hadron Collider.

\section*{Select talks}
\years{2024}NIR: A unified instruction set for brain-inspired computing --- \href{https://sites.google.com/view/telluride-2024/}{Telluride Neuromorphic Cognition Engineering Workshop} and \href{http://snufa.net/}{Spiking Neural networks as Universal Function Approximators} \\
\years{2023}AEStream: Accelerated event-based processing with coroutines --- \href{https://sites.google.com/view/telluride-2023/}{Telluride Neuromorphic Cognition Engineering Workshop}. \\
\years{2023}Translation and Scale Invariance for Event-Based Object tracking ---
\href{https://niceworkshop.org/}{Neuro Inspired Computational Elements Conference (NICE)}. \\
\years{2023}The need for neuromorphic abstractions ---
\href{https://open-neuromorphic.org/}{Open Neuromorphic workshop}. \\
\years{2021}Norse: A library for gradient-based learning in Spiking Neural Networks --- Workshop on \href{http://snufa.net/}{Spiking Neural networks as Universal Function Approximators (SNUFA)}.

\section*{Selected student projects}
\years{2024}Oskar Strömberg, \emph{Event-based vision with spiking vision transformers}, KTH, MSc. \\
\years{2022}Philpp Mondorff, \emph{Spiking Reinforcement Learning for Robust Robot Control}, KTH, MSc.\\
\years{2022}Merlin Sewina, \emph{Decoding EEG with Spiking and Artificial Neural Networks}, KTH, MSc.\\
\years{2020}Mikkel Ziemer, \emph{Building CERN's Control System}, CPHBusiness, BSc.

\vspace{-0.2cm}
\section*{Contributions to the community}
I review for several neuromorphic venues, including the \href{https://iopscience.iop.org/journal/2634-4386}{Journal of Neuromorphic Computing and Engineering}, the \href{https://niceworkshop.org/}{Neuro Inspired Computational Elements Conference (NICE)}, and the \href{https://iconsneuromorphic.cc/}{International Conference on Neuromorphic Systems (ICONS)}.
I have organized and led numerous events including at the \href{https://sites.google.com/view/telluride-2024/}{Telluride Neuromorphic Cognition Engineering Workshop} and \href{https://capocaccia.cc/en/}{CapoCaccia Workshop toward Neuromorphic Intelligence}.
Finally, I am adamant in my support and maintenance of open-source efforts.
I am an active part of the \emph{\href{https://open-neuromorphic.org/}{Neuromorphic Computing and Engineering Community}} and I am
driving development of \href{https://github.com/aestream/faery}{Faery}, \href{https://github.com/aestream}{AEstream}, and \href{https://github.com/norse/}{Norse}.
%\years{2024}\textbf{Topic area co-organizer}: ``Neuromorphic systems for space applications''---\emph{} \\
%\years{2022}\textbf{Real-time neuromorphics with Norse and AEStream}---\emph{\href{https://jepedersen.dk/slides/202209_neurotech/index.html}{Neurotech Tutorial}} \\
%\years{2020-2023}\textbf{Conference committee member}---5th, 6th, and 7th Human Brain Project Student Conference on on Interdisciplinary Brain Research.\\
% \years{2020-2023}\textbf{Student ambassador}---Human Brain Project.\\
%\years{2024-}\textbf{Faery}---a library for processing event-based data (\href{https://github.com/aestream}{github.com/aestream}) \\
%\years{2020-2024}\textbf{AEStream}---a coroutines-based library for processing event-based data (\href{https://github.com/aestream}{github.com/aestream}) \\
%\years{2018-}\textbf{Norse}---a PyTorch-based deep learning library for spiking neural networks (\href{https://github.com/norse/}{github.com/norse}).
\vspace{-0.2cm}
\section*{Technical training}

\textbf{Teaching}: Supervision, Communication, and Teaching. \\
\textbf{Courseworks}: Advanced machine learning, Deep neural networks, Computational Neuroscience, Advanced Computer vision. \\
\textbf{Programming}: Python, C++, C, Rust, CUDA, PyTorch, JAX, NumPy, SciPy. \\
\textbf{Software Engineering}: Git, Docker, Linux, DevOps, NixOS, CI/CD, Testing, Documentation. \\

\begin{comment}
\section*{Technical training}

\years{2018}``The Brain Simulation Platform of the Human Brain
Project'', Human Brain Project summer school  --- \emph{Palermo, Italy}\\
\years{2018}``Transdisciplinary
Research Linking Neuroscience, Brain Medicine and Computer Science'', Human Brain Project
young researchers event ---
\emph{Ljubljana, Slovenia}\\
\years{2017}2\textsuperscript{nd} Young Researchers Event, Human Brain Project ---
\emph{Geneva, Switzerland}\\
\years{2014}AXEL course on particle accelerator physics --- \emph{CERN, Switzerland}

\section*{Voluntary work}
\years{2014-2016} Lecturer and tour guide for visitors in and around the CERN complex.
I was given in-depth introductions to the different parts of the
control system in the accelerator complex, the cryogenics
facilities and particle physics. \\
\years{2008-2015}Co-creator and lead programmer of the collaborative drawing-tool RepoCad, now published as open-source
(\href{https://github.com/selftiesoftware}{github.com/selftiesoftware}).

\section*{Languages}
\emph{Danish} (C2, native)\\
\emph{English} (C2, fully proficient)\\
\emph{French} (B1, working basic conversational) \\
\emph{German} (B1, working basic conversational)

\pagebreak
\section*{References}
The following professionals kindly offered their availability for personal
references and questions:
\begin{enumerate}
      \item Jesper Mogensen, Professor, Unit for Cognitive Neuroscience, Copenhagen University
            \\ Email: \texttt{jesper.mogensen@psy.ku.dk}
            \\ Phone: \texttt{(+45) 3532 4873}
      \item Lars Bogetoft, Head of IT Program, Copenhagen Business Academy (former)
            \\ Email: \texttt{lars.bogetoft@gmail.com}
            \\ Phone: \texttt{(+45) 5185 0497}
      \item Kevin Lee, CEO, Mobilized Construction
            \\ Email: \texttt{kevin@mobilizedconstruction.com}
            \\ Phone: \texttt{(+44) 794 046 3174}
\end{enumerate}
\end{comment}

\end{document}
