\documentclass[12pt,a4paper,notitlepage]{article}
\usepackage[utf8]{inputenc}
\usepackage{amsmath}
\usepackage{amsfonts}
\usepackage{amssymb}
\usepackage{tabularx}

\usepackage{titling}
\setlength{\droptitle}{-10em} 

\author{Jens Egholm Pedersen, born 1988 
\\ \texttt{jens.egholm@cern.ch}
}
\title{Curriculum Vitae}
\begin{document}
\maketitle
\section*{Education}
\begin{tabularx}{\textwidth}{l X}
2013 - 2013 & ERASMUS exchange programme, spring semester 2013, \\
            & Ècole d'Ingenieurs en Informatique, EPITA Paris, France. \\
2011 - 2015 & BSc. in software development, IT-University of Copenhagen, Denmark. \\
2008 - 2011 & BSc. in political science, Aarhus University, Denmark. \\
2004 - 2007 & High school, Esbjerg Gymnasium \& HF, Denmark.
\end{tabularx}

\section*{Employments}
\begin{tabularx}{\textwidth}{l X}
2014 - 2015 & Software engineer at CERN. \\
            & At CERN I have been driving the development and maintenance of a database 
of controls data and machine logs, based on Elasticsearch and Kibana, with a transport layer based on SYSLOG, TCP and JMS.
I worked mainly in Java 8 but have installed, maintained and monitored the data cluster on several linux hosts. \\
            & Second, I have been involved in the creation of a testbed in Java for one of the central control systems
for the LHC accelerator. As a part of this project I wrote my bachelor thesis on predictable real-time guarantees in the JVM. \\
2013 - 2013 & Teaching assistant in a course on Mobile and Distributed Computing, IT-University of Copenhagen. \\
2012 - 2013 & Student programmer at the DemTech research group. \\
2011 - 2013 & Student ambassador for the IT-University of Copenhagen.
\end{tabularx}

\section*{Volunteer work}
\begin{tabularx}{\textwidth}{l X}
2014 - 2016 & Tour guide for visitors around the CERN complex. \\
2011 - 2011 & Project lead, Ignition-Point entertainment. \\
2010 - 2011 & Leading sound engineer, Ignition-Point entertainment. \\
2010 - 2011 & Employee at the study café Lektier-Online, State and University  Library of Aarhus.  \\
2008 - Current & Co-creator and lead programmer of the collaborative drawing-tool RepoCad (\texttt{http://repocad.com/}).
%2006 - 2007 & Member of the board of the Danish High School Association (DGS). \\
%2005 - 2007 & Chairman of the student council, Esbjerg Gymnasium \& HF. \\
\end{tabularx}

\section*{Papers}
\textit{Introducing RepoCad  - A prototype of the Internet of Digital Design}, eCAADe 33, Vienna 2015. \\
\textit{Predictable firm real-time Java}. Bachelor thesis, IT-University of Copenhagen, 2015. \\
\textit{Proxy voting in the European Union}. Bachelor thesis, Aarhus University, 2011.

\end{document}
