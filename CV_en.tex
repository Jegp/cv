%------------------------------------
% Dario Taraborelli
% Typesetting your academic CV in LaTeX
%
% URL: http://nitens.org/taraborelli/cvtex
% DISCLAIMER: This template is provided for free and without any guarantee
% that it will correctly compile on your system if you have a non-standard
% configuration.
% Some rights reserved: http://creativecommons.org/licenses/by-sa/3.0/
%------------------------------------

%!TEX TS-program = xelatex
%!TEX encoding = UTF-8 Unicode

\documentclass[11pt, a4paper]{article}
\usepackage{fontspec}
\usepackage{amssymb}

% DOCUMENT LAYOUT
\usepackage{geometry}
\geometry{a4paper, textwidth=5.8in, textheight=10in, marginparsep=7pt, marginparwidth=.6in}
\setlength\parindent{0in}

% COMMENTS
\usepackage{verbatim}

% FONTS
\usepackage[usenames,dvipsnames]{xcolor}
\usepackage{xunicode}
\usepackage{xltxtra}
\defaultfontfeatures{Mapping=tex-text}
%\setromanfont [Ligatures={Common}, Numbers={OldStyle}, Variant=01]{Linux Libertine O}
%\setmonofont[Scale=0.8]{Monaco}
%%% modified by Karol Kozioł for ShareLaTeX use
\setmainfont[
  Ligatures={Common}, Numbers={OldStyle}, Variant=01,
  BoldFont=LinLibertine_RB.otf,
  ItalicFont=LinLibertine_RI.otf,
  BoldItalicFont=LinLibertine_RBI.otf
]{LinLibertine_R.otf}
\setmonofont[Scale=0.8]{DejaVuSansMono.ttf}

% ---- CUSTOM COMMANDS
\chardef\&="E050
\newcommand{\html}[1]{\href{#1}{\scriptsize\textsc{[html]}}}
\newcommand{\pdf}[1]{\href{#1}{\scriptsize\textsc{[pdf]}}}
\newcommand{\doi}[1]{\href{#1}{\scriptsize\textsc{[doi]}}}
% ---- MARGIN YEARSWho?	Uni	City	State
\usepackage{marginnote}
\newcommand{\amper{}}{\chardef\amper="E0BD }
\newcommand{\years}[1]{\marginnote{\scriptsize #1}}
\renewcommand*{\raggedleftmarginnote}{}
\setlength{\marginparsep}{7pt}
\reversemarginpar

\usepackage[symbol]{footmisc}

% HEADINGS
\usepackage{sectsty}
\usepackage[normalem]{ulem}
\sectionfont{\mdseries\upshape\Large}
\subsectionfont{\mdseries\scshape\normalsize}
\subsubsectionfont{\mdseries\upshape\large}

\usepackage{titlesec}
\titlespacing*{\section} {0pt}{3ex plus 1ex minus .2ex}{1.3ex plus .2ex}
\titlespacing*{\subsection} {0pt}{2.25ex plus 1ex minus .2ex}{0.5ex plus .2ex}

% PDF SETUP
\usepackage[%dvipdfm,
bookmarks, colorlinks, breaklinks,
	pdftitle={Jens Egholm Pedersen - curriculum vitae},
	pdfauthor={Jens Egholm Pedersen},
	pdfproducer={https://jepedersen.dk}
]{hyperref}
\hypersetup{linkcolor=blue,citecolor=blue,filecolor=black,urlcolor=MidnightBlue}
\begin{document}

{\LARGE Jens Egholm Pedersen} \hspace{1.4cm} {\large Born: 1988, Danish citizenship}\\[.3cm]
Contact: \href{mailto:jeped@kth.se}{jeped@kth.se}\hspace{0.4cm}
Website: \href{https://jepedersen.dk}{jepedersen.dk} \hspace{0.4cm}
GitHub: \href{https://github.com/jegp/}{@jegp}\hspace{0.4cm}
\href{https://orcid.org/0000-0001-6012-7415}{ORCID}


\section*{Education}
\years{2019-2025\\{Sep. (est)}}\textbf{PhD in Computer Science}, KTH Royal Institute of Technology, Sweden. \\
Thesis: \textbf{Neuromorphic computing in space and time}.\\
Advisors:
\href{https://www.kth.se/profile/conr}{Jörg Conradt} (KTH) and
\href{https://www.kth.se/profile/arvindku}{Arvind Kumar} (KTH, Karolinska Institute). \\
Research stay: Stanford University, supervised by \href{https://profiles.stanford.edu/sadasivan-shankar}{Sadasivan Shankar}.\\
\years{2016-2019}\textbf{MSc in IT} \& \textbf{Cognition}, University of Copenhagen.\\
Thesis: Modelling learning systems in spiking and artificial neural networks.
\\
\years{2011-2015}\textbf{BSc in Computer Science}, IT-University of Copenhagen.\\
Thesis: Predictable firm real-time Java. In collaboration with CERN.\\
\years{2009-2011}\textbf{BSc in Political Science}, University of Aarhus, Denmark.\\
Thesis: Proxy voting in the European Union.

\section*{Honours, Grants \& awards}
\years{2025}\href{https://kaw.wallenberg.org/en/calls/wallenberg-foundation-postdoctoral-scholarship-stanford-university-usa}{Wallenberg-Bienenstock Postdoctoral Fellowship Program} at Stanford University, USA.\\
\years{2025}\href{https://www.mahowaldprize.org/prize-awards/prizes-2025}{Mahowald Early Career Award} for work on the Neuromorphic Intermediate Representation.\\
\years{2024}\href{https://www.neuropac.info/fellowships/}{NSF AccelNet NeuroPAC Fellowship} at Stanford University, USA.\\
\years{2022 - 2025}Several compute grants for Swedish National Infrastructure.

\section*{Peer-reviewed publications {\small (* indicates equal authorship)}}
\years{2025}G. D'Angelo*, \textbf{J. E. Pedersen}*, T. Hassan, M. Cianchetti, J. Bongard, F. Iida, G. Indiveri, M. Hoffmann, C. Laschi, C. D. Luca, C Bartolozzi, E. Donati. ``A Benchmarking Framework for Embodied Neuromorphic Agents''. \textit{In review}. \\
\years{2025}\textbf{J. E. Pedersen}, J. Conradt, \& T. Lindeberg. ``\href{https://arxiv.org/abs/2405.00318}{Covariant spatio-temporal receptive fields for spiking neural networks}''. \href{https://www.nature.com/ncomms/}{Nature Communications} [IF: 17.69]. \\
\years{2025}J. P. Romero B., D. Korakovounis, \textbf{J. E. Pedersen}, J. Conradt. ``\href{https://doi.org/10.1088/2634-4386/addc15}{Low-latency neuromorphic air hockey player}''. \href{https://iopscience.iop.org/journal/2634-4386}{Neuromorphic Computing and Engineering} [IF: 6.1]. \\
\years{2025} P. Taborsky, I. Colonnelli, K. Kurowski, R. Sarma, N. H. Pontoppidan, B. Jansík, N. S. Detlefsen, \textbf{J. E. Pedersen}, R. Larsen, L. K. Hansen. ``\href{https://www.sciencedirect.com/science/article/pii/S1877050925006301}{Towards a European HPC/AI ecosystem: a community-driven report}.'' \href{https://www.sciencedirect.com/journal/procedia-computer-science/vol/255/suppl/C}{Proceedings of the Second EuroHPC user day}. \\
\years{2024}\textbf{J. E. Pedersen}*, S. Abreau*, J. Eshraghian, et al. ``\href{https://doi.org/10.1038/s41467-024-52259-9}{Neuromorphic Intermediate Representation}''. \href{https://www.nature.com/ncomms/}{Nature Communications} [IF: 17.69]. \\
\years{2024} S. Abreau*, \textbf{J. E. Pedersen}*, K. Heckel*, \& A. Pierro. ``\href{https://openreview.net/forum?id=30Rq6yXr8I}{Q-S5: Towards Quantized State Space Models}.'' \href{https://icml.cc/virtual/2024/workshop/29962}{ICML - Next Generation of Sequence Modeling Architectures} [Acceptance rate: 30.5\%]. \\
\years{2024} S. Abreau, \textbf{J. E. Pedersen}.
``\href{https://ieeexplore.ieee.org/document/10766507}{Neuromorphic Programming: Emerging Directions for Brain-Inspired Hardware}''.
\href{https://iconsneuromorphic.cc/}{International Conference on Neuromorphic Computing Systems}. \\
\years{2024}A. Geminiani, J. Kathrein, ..., \textbf{J. E. Pedersen} et al. ``\href{https://link.springer.com/article/10.1007/s12021-024-09682-6}{Multidisciplinary and collaborative training in neuroscience: Insights from the Human Brain Project Education Programme}''. \href{https://link.springer.com/journal/12021}{Neuroinformatics} [IF: 2.7]. \\
\years{2023} \textbf{J. E. Pedersen}, R. Singhal, \& J. Conradt.
``\href{https://dl.acm.org/doi/10.1145/3584954.3584996}{Translation and Scale Invariance for Event-Based Object tracking}''.
\href{https://niceworkshop.org/}{Neuro Inspired Computational Elements Conference (NICE)}. \\
\years{2023} \textbf{J. E. Pedersen} \& J. Conradt.
``\href{https://dl.acm.org/doi/fullHtml/10.1145/3584954.3584997}{AEStream: Accelerated event-based processing with coroutines}.''
\href{https://niceworkshop.org/}{Neuro Inspired Computational Elements Conference (NICE)}. \\
\years{2023} J. P. Romero B., L. A. Plana, A. Rowley, M. Hessel, \textbf{J. E. Pedersen}, S. Furber, J. Conradt.
``\href{https://dl.acm.org/doi/pdf/10.1145/3589737.3605969}{A High-Throughput Low-Latency Interface Board for SpiNNaker-in-the-loop Real-Time Systems.}''
\href{https://icons.ornl.gov/}{ICONS - International Conference on Neuromorphic Systems}. \\
\years{2018} J. Mogensen, N. Dauggaard, S. Kitsios, \textbf{J. E. Pedersen}, M. Overgaard.
``\href{https://researchprofiles.ku.dk/en/publications/understanding-the-neurocognitive-organization-as-strategies-rathe}{Understanding the neurocognitive organization as strategies rather than functions: Implications for neurological research}.''
\href{https://ecronicon.net/ec_neurology}{EC Neurology}.

\section*{Open-access publications {\small (* indicates equal authorship)}}
\years{2025} \textbf{J. E. Pedersen}, D. Korakovounis, J. Conradt, ``\href{https://arxiv.org/abs/2412.03259}{GERD: Geometric event response data generation}.'' \href{https://arxiv.org/abs/2412.03259}{arXiv}. \\
\years{2022} J. Turner, \textbf{J. E. Pedersen}, J. Conradt, \& T Nowotny, ``\href{https://dl.acm.org/doi/10.1145/3517343.3517378}{Event-based dataset for classification and pose estimation}.''
\href{https://niceworkshop.org/}{Neuro Inspired Computational Elements Conference (NICE)}. \\
\years{2020} \textbf{J. E. Pedersen}*, C. Pehle*, ``\href{https://zenodo.org/record/4422025}{Norse - Spiking neural network for deep learning}.''
\href{https://zenodo.org}{Zenodo}.

\section*{Professional Experience}
\years{2018-2025}\textbf{External lecturer}, IT-University of Copenhagen, Denmark.
I planned and taught courses on Python and data science to 500+ students with outstanding reviews ($\geqslant$ 5.2/6.0). \\
\years{2016-2019}\textbf{Adjunct professor}, Copenhagen Business Academy,
Denmark.
I lectured on machine learning, artificial intelligence, business analytics and distributed systems infrastructure to 150+ students with outstanding reviews.\\
\years{2016-2019}\textbf{Chief Technology Officer}, Mobilized Construction, Denmark, Kenya.
I designed a globally distributed software stack, managed 15+ team members in US, Wales, Kazakhstan, and Kenya.\\
\years{2014-2016}\textbf{Software engineer}, CERN, Switzerland.
I developed and maintained critical monitoring tool-chain and testbed infrastructure for the Large Hadron Collider.

\section*{Selected talks}
\years{2025}Open-Source Neuromorphic Research Infrastructure --- \href{https://open-neuromorphic.org/workshops/}{Open Neuromorphic Workshop}. \\
\years{2025}Reproducibility and interoperability in neuromorphic computing --- \href{https://www.nengo.ai/summer-school/}{Nengo Summer School}. \\
\years{2024}NIR: A unified instruction set for brain-inspired computing --- \href{https://sites.google.com/view/telluride-2024/}{Telluride Neuromorphic Cognition Engineering Workshop} and \href{http://snufa.net/}{Spiking Neural networks as Universal Function Approximators} \\
\years{2023}Translation and Scale Invariance for Event-Based Object tracking ---
\href{https://niceworkshop.org/}{Neuro Inspired Computational Elements Conference (NICE)}. \\
\years{2023}The need for neuromorphic abstractions ---
\href{https://open-neuromorphic.org/}{Open Neuromorphic workshop}. \\
\years{2021}Norse: A library for gradient-based learning in Spiking Neural Networks --- Workshop on \href{http://snufa.net/}{Spiking Neural networks as Universal Function Approximators (SNUFA)}.

\section*{Supervision experience}
I supervised 10+ students during my work as a PhD student at KTH, adjunct professor at Copenhagen Business Academy, and CTO for Mobilized Construction.
They resulted in multiple community contributions and published work.
Selected projects are listed below. \\
\years{2024}Oskar Strömberg, \emph{Event-based vision with spiking vision transformers}, KTH, MSc. \\
\years{2022}Philpp Mondorff, \emph{Spiking Reinforcement Learning for Robust Robot Control}, KTH, MSc.\\
\years{2022}Merlin Sewina, \emph{Decoding EEG with Spiking and Artificial Neural Networks}, KTH, MSc.\\
\years{2020}Mikkel Ziemer, \emph{Building CERN's Control System}, CPHBusiness, BSc.

\vspace{-0.2cm}
\section*{Community Contributions}
I am driving development of several widely-used neuromorphic computing libraries (\href{https://github.com/neuromorphs/NIR}{NIR}, \href{https://github.com/norse/}{Norse}, \href{https://github.com/aestream}{AEStream}, \href{https://github.com/aestream/faery}{Faery}) with over 500,000 downloads.
I review for several neuromorphic venues, including the \href{https://iopscience.iop.org/journal/2634-4386}{Journal of Neuromorphic Computing and Engineering}, the \href{https://niceworkshop.org/}{Neuro Inspired Computational Elements Conference (NICE)}, and the \href{https://iconsneuromorphic.cc/}{International Conference on Neuromorphic Systems (ICONS)}.
I have organized and led numerous events including at the \href{https://sites.google.com/view/telluride-2024/}{Telluride Neuromorphic Cognition Engineering Workshop} and \href{https://capocaccia.cc/en/}{CapoCaccia Workshop toward Neuromorphic Intelligence}.
Finally, I am chairing the \emph{\href{https://open-neuromorphic.org/}{Open Neuromorphic Community}}.
%\years{2024}\textbf{Topic area co-organizer}: ``Neuromorphic systems for space applications''---\emph{} \\
%\years{2022}\textbf{Real-time neuromorphics with Norse and AEStream}---\emph{\href{https://jepedersen.dk/slides/202209_neurotech/index.html}{Neurotech Tutorial}} \\
%\years{2020-2023}\textbf{Conference committee member}---5th, 6th, and 7th Human Brain Project Student Conference on on Interdisciplinary Brain Research.\\
% \years{2020-2023}\textbf{Student ambassador}---Human Brain Project.\\
%\years{2024-}\textbf{Faery}---a library for processing event-based data (\href{https://github.com/aestream}{github.com/aestream}) \\
%\years{2020-2024}\textbf{AEStream}---a coroutines-based library for processing event-based data (\href{https://github.com/aestream}{github.com/aestream}) \\
%\years{2018-}\textbf{Norse}---a PyTorch-based deep learning library for spiking neural networks (\href{https://github.com/norse/}{github.com/norse}).
% \vspace{-0.2cm}
\begin{comment}
\section*{Technical training}

\textbf{Teaching}: Supervision, Communication, and Teaching.\\
\textbf{Courseworks}: Advanced machine learning, Deep neural networks, Computational % Neuroscience, Advanced Computer vision.\\
\textbf{Programming}: Python, C++, C, Rust, CUDA, PyTorch, JAX, NumPy, SciPy.\\
\textbf{Software Engineering}: Git, Docker, Linux, DevOps, NixOS, CI/CD, Testing, Documentation.

\section*{Technical training}

\years{2018}``The Brain Simulation Platform of the Human Brain
Project'', Human Brain Project summer school  --- \emph{Palermo, Italy}\\
\years{2018}``Transdisciplinary
Research Linking Neuroscience, Brain Medicine and Computer Science'', Human Brain Project
young researchers event ---
\emph{Ljubljana, Slovenia}\\
\years{2017}2\textsuperscript{nd} Young Researchers Event, Human Brain Project ---
\emph{Geneva, Switzerland}\\
\years{2014}AXEL course on particle accelerator physics --- \emph{CERN, Switzerland}

\section*{Voluntary work}
\years{2014-2016} Lecturer and tour guide for visitors in and around the CERN complex.
I was given in-depth introductions to the different parts of the
control system in the accelerator complex, the cryogenics
facilities and particle physics. \\
\years{2008-2015}Co-creator and lead programmer of the collaborative drawing-tool RepoCad, now published as open-source
(\href{https://github.com/selftiesoftware}{github.com/selftiesoftware}).

\section*{Languages}
\emph{Danish} (C2, native)\\
\emph{English} (C2, fully proficient)\\
\emph{French} (B1, working basic conversational) \\
\emph{German} (B1, working basic conversational)

\pagebreak
\section*{References}
The following professionals kindly offered their availability for personal
references and questions:
\begin{enumerate}
	\item Jesper Mogensen, Professor, Unit for Cognitive Neuroscience, Copenhagen University
	      \\ Email: \texttt{jesper.mogensen@psy.ku.dk}
	      \\ Phone: \texttt{(+45) 3532 4873}
	\item Lars Bogetoft, Head of IT Program, Copenhagen Business Academy (former)
	      \\ Email: \texttt{lars.bogetoft@gmail.com}
	      \\ Phone: \texttt{(+45) 5185 0497}
	\item Kevin Lee, CEO, Mobilized Construction
	      \\ Email: \texttt{kevin@mobilizedconstruction.com}
	      \\ Phone: \texttt{(+44) 794 046 3174}
\end{enumerate}
\end{comment}

\end{document}
