%------------------------------------
% Dario Taraborelli
% Typesetting your academic CV in LaTeX
%
% URL: http://nitens.org/taraborelli/cvtex
% DISCLAIMER: This template is provided for free and without any guarantee 
% that it will correctly compile on your system if you have a non-standard  
% configuration.
% Some rights reserved: http://creativecommons.org/licenses/by-sa/3.0/
%------------------------------------

%!TEX TS-program = xelatex
%!TEX encoding = UTF-8 Unicode

\documentclass[11pt, a4paper]{article}
\usepackage{fontspec} 

% DOCUMENT LAYOUT
\usepackage{geometry} 
\geometry{a4paper, textwidth=5.5in, textheight=8.5in, marginparsep=7pt, marginparwidth=.6in}
\setlength\parindent{0in}

% COMMENTS
\usepackage{verbatim}

% FONTS
\usepackage[usenames,dvipsnames]{xcolor}
\usepackage{xunicode}
\usepackage{xltxtra}
\defaultfontfeatures{Mapping=tex-text}
%\setromanfont [Ligatures={Common}, Numbers={OldStyle}, Variant=01]{Linux Libertine O}
%\setmonofont[Scale=0.8]{Monaco}
%%% modified by Karol Kozioł for ShareLaTeX use
\setmainfont[
  Ligatures={Common}, Numbers={OldStyle}, Variant=01,
  BoldFont=LinLibertine_RB.otf,
  ItalicFont=LinLibertine_RI.otf,
  BoldItalicFont=LinLibertine_RBI.otf
]{LinLibertine_R.otf}
\setmonofont[Scale=0.8]{DejaVuSansMono.ttf}

% ---- CUSTOM COMMANDS
\chardef\&="E050
\newcommand{\html}[1]{\href{#1}{\scriptsize\textsc{[html]}}}
\newcommand{\pdf}[1]{\href{#1}{\scriptsize\textsc{[pdf]}}}
\newcommand{\doi}[1]{\href{#1}{\scriptsize\textsc{[doi]}}}
% ---- MARGIN YEARS
\usepackage{marginnote}
\newcommand{\amper{}}{\chardef\amper="E0BD }
\newcommand{\years}[1]{\marginnote{\scriptsize #1}}
\renewcommand*{\raggedleftmarginnote}{}
\setlength{\marginparsep}{7pt}
\reversemarginpar

% HEADINGS
\usepackage{sectsty} 
\usepackage[normalem]{ulem} 
\sectionfont{\mdseries\upshape\Large}
\subsectionfont{\mdseries\scshape\normalsize} 
\subsubsectionfont{\mdseries\upshape\large} 

% PDF SETUP
\usepackage[%dvipdfm, 
bookmarks, colorlinks, breaklinks, 
	pdftitle={Jens Egholm Pedersen - curriculum vitae},
	pdfauthor={Jens Egholm Pedersen},
	pdfproducer={https://github.com/jegp}
]{hyperref}  
\hypersetup{linkcolor=blue,citecolor=blue,filecolor=black,urlcolor=MidnightBlue} 

\begin{document}

{\LARGE Jens Egholm Pedersen}\\[1cm]
  Schacksgade 14, 4. th\\
  1365 Copenhagen K, Denmark\\[.2cm]
Phone: +45 25122752\\
Email: \href{mailto:jegp@itu.dk}{jegp@itu.dk}\\

Born: 8. August 1988---Varde, Denmark\\
Nationality: Danish

\section*{Education}
\noindent
\years{2019}\textbf{MSc in Computer Science}, University of Copenhagen\\
Thesis: \textbf{Modelling learning systems in spiking and artificial neural networks}\\
\noindent
\begin{tabular}{@{}r @{\hspace{0.1cm}} l}
\emph{Advisors:} & \emph{Martin Elsman (Department of Computer Science) and}\\
   & \emph{Jesper Mogensen (Unit for Cognitive Neuroscience, co-supervisor)}
\end{tabular}\\
\years{2015}\textbf{BSc in Computer Science}, IT-University of Copenhagen\\
Thesis: \textbf{Predictable firm real-time Java}\\
\noindent
\begin{tabular}{@{}r @{\hspace{0.1cm}} l}
\emph{Advisors:} & \emph{Peter Sestoft (IT-University of Copenhagen) and}\\
   & \emph{Vito Baggiolini (CERN)}
\end{tabular}\\
\years{2011}\textbf{BSc in Political Science}, University of Aarhus\\
Thesis: \textbf{Proxy voting in the European Union}\\
\emph{Advisor: Tore V. Olsen (Department of Political Science)}

\section*{Appointments held}
\noindent
\years{2018-}\textbf{External lecturer}, IT-University of Copenhagen, Denmark\\
At ITU I have planned and taught an introduction seminar for master's students
on programming, programming tools and computational literacy with outstanding reviews. \\[0.5cm]
\years{2016-}\textbf{Adjunct professor}, Copenhagen Business Academy,
Denmark\\
As an adjunct professor I developed materials and lectured in
a range of topics, including machine learning, artificial intelligence, business
analytics and design and operations of large systems.
I received high praise for my ability to internalise and communicate academic
material, as well as my level of dedication and pedagogical skills.\\[0.5cm]
\years{2016-}\textbf{Chief Technology Officer}, Mobilized Construction, Denmark, Kenya\\
At Mobilized Construction I designed and implemented infrastructure required for
their globally distributed infrastructure.
My role also involved managerial responsibilities for our development teams in 
Wales and Kenya, and I personally supervised student
projects at the master's level from the University of Southern
California and the University of Wales.\\[0.5cm]
\years{2014-2016}\textbf{Software engineer}, CERN, Switzerland\\
At CERN I was driving the development and maintenance of
a monitoring toolchain for the accelerator control system complex.
My contract was extended to work on a testbed for the Large Hadron
Collider (LHC) accelerator, with an emphasis on real-time critical
systems architecture in Java. \\[0.5cm]
\years{2013}\textbf{Instructor}, Mobile and Distributed Systems, IT-University of
Copenhagen, Denmark\\
I lectured on principles and research on mobile and distributed
systems and participated in the development of new teaching
materials. \\[0.5cm]
\years{2012-2014}\textbf{Research assistant}, DemTech research group, Denmark\\
As a student programmer I developed a 3G based router system to
measure the length of election queues. Used for nation-wide
research and deployed at several polling places in the Danish national
election of 2013.


\section*{Publications and talks}
\noindent

\subsection*{Peer-reviewed journal articles}
\years{2018}\textbf{Understanding the neurocognitive organization as strategies rather
than functions: Implications for neurological research}\\
Mogensen, J., Dauggaard, N., Kitsios, S., Pedersen, J. E., Overgaard, M.
\emph{EC Neurology}

\subsection*{Peer-reviewed conference papers}
\years{2015}\textbf{Introducing RepoCad - A prototype of the Internet of Digital
Design}\\
Jackson, O. E. and Pedersen, J. E.\\
\emph{eCAADe}, Volume 33

\subsection*{Peer-reviewed conference abstracts and posters}
\years{2019}\textbf{Volr - A declarative approach to learning in spiking neural networks} 
--- \href{https://education.humanbrainproject.eu/web/3rd-hbp-student-conference}{NEST
conference}, abstract, poster\\
Pedersen, J. E. and Pehle, Christian,\\
\emph{3rd HBP Student Conference On Interdisciplinary Brain Research}

\years{2018}\textbf{Volr: A declarative interface language for neural computation}
--- \href{https://indico-jsc.fz-juelich.de/event/71/}{NEST
conference}, abstract, poster\\
Pedersen, J. E. and Pehle, Christian,\\
\emph{NEST conference: A Forum for Users and Developers}

\subsection*{Conference presentations}
\years{2018}\textbf{Experimental neural systems modelling with Jupyter
Notebooks} ---
\emph{HBP School} \\
Pedersen, J. E. \\[0.4cm]
\years{2018}\textbf{Volr: A declarative interface language for neural
computation}
--- \emph{NEST conference}\\
Pedersen, J. E. and Pehle, C. 


\section*{Technical training}

\years{2018}``The Brain Simulation Platform of the Human Brain
Project'', Human Brain Project summer school  --- \emph{Palermo, Italy}\\
\years{2018}``Transdiscilinary
Research Linking Neuroscience, Brain Medicine and Computer Science'', Human Brain Project
young researchers event ---
\emph{Ljubljana, Slovenia}\\
\years{2017}2\textsuperscript{nd} Young Researchers Event, Human Brain Project ---
\emph{Geneva, Switzerland}\\
\years{2014}AXEL course on particle accelerator physics --- \emph{CERN, Switzerland}

\section*{Languages}
\emph{Danish} (C2, native)\\
\emph{English} (C2, fully proficient)\\
\emph{French} (B1, working basic conversational) \\
\emph{German} (B1, working basic conversational)

\section*{Honours \& awards}
\noindent
\years{2017}Finalist, Founders of Tomorrow
(\href{https://foundersoftomorrow.com}{FoundersOfTomorrow.com})\\
\years{2013}ERASMUS exchange programme, Ècole d'Ingenieurs en Informatique, Paris, France\\

\section*{Voluntary work}
\noindent
\years{2018-}Lead developer and maintainer of Volr, a Haskell-based environment
for the construction, evaluation and analysis of neural networks
(\href{https://github.com/volr/}{github.com/volr}). \\
\years{2014-2016} Lecturer and tour guide for visitors in and around the CERN complex.
I was given in-depth introductions to the different parts of the
control system in the accelerator complex, the cryogenics
facilities and particle physics. \\
\years{2008-2015}Co-creator and lead programmer of the collaborative drawing-tool RepoCad, now published as open-source
(\href{https://github.com/selftiesoftware}{github.com/selftiesoftware}).

\section*{Hobbies}
\years{\centering \large \textbullet}Cognitive science and philosophy of the mind. Has a personal goal to augment his cognitive capacities.\\
\years{\centering \large \textbullet}Volunteer work. Believes in a social responsibility for the immediate community and society as a whole.\\
\years{\centering \large \textbullet}Device hacking. Routinely programs devices around him to support daily routines.\\
\years{\centering \large \textbullet}Music. Started playing the violin since age 6 and piano from age 12.\\
\years{\centering \large \textbullet}Hiking. Primarily in the Alps, Norway and Sweden.

\end{document}
