\documentclass[12pt,a4paper,notitlepage]{article}
\usepackage[utf8]{inputenc}
\usepackage{amsmath}
\usepackage{amsfonts}
\usepackage{amssymb}
\usepackage{tabularx}

\usepackage{fancyhdr}
\usepackage{lastpage}
\pagestyle{fancy}
\fancyhf{}

\usepackage{geometry}
\newgeometry{margin=2cm}

\renewcommand{\headrulewidth}{0pt}
\renewcommand{\footrulewidth}{0pt}

\usepackage{titling}
\setlength{\droptitle}{-8em}
\cfoot{Page \thepage \hspace{1pt} of \pageref{LastPage}}

\author{Jens Egholm Pedersen, born 1988
\\ \texttt{jensegholm@protonmail.com}
}

\makeatletter
\let\ps@plain\ps@fancy
\makeatother

\title{Curriculum Vitae}
\begin{document}
\maketitle

\section*{Education}
\begin{tabularx}{\textwidth}{l X}
2016 - 2018 & Student, MSc IT \& Cognition, Copenhagen University, Denmark \\
            & \footnotesize Graduation 23rd of October 2018 \\
2014 - 2014 & Technical training on particle accelerator physics \\
2013 - 2013 & ERASMUS exchange programme, spring semester 2013, \\
            & Ècole d'Ingenieurs en Informatique, EPITA Paris, France \\
2011 - 2015 & BSc. in software development, IT-University of Copenhagen, Denmark \\
2008 - 2011 & BSc. in political science, Aarhus University, Denmark
\end{tabularx}

\section*{Employments}
\begin{tabularx}{\textwidth}{l X}
2018 -      & \textbf{External lecturer, IT-University of Copenhagen} \\
            & Planned and executed a qualification seminar for master's students,
              beginning August 2018. \\
2016 - 2018 & \textbf{Assistant professor, Copenhagen Business Academy}\footnotemark[1] \\
            & As an assistant professor I developed materials and lectured in
              a range of topics, including functional programming,
              machine learning and design and operation of large systems.
              I received high praise for the professional content as well as
              my communication and pedagogical skills.\\
2016 - 2018 & \textbf{CTO and co-founder, Mobilized Construction}\footnotemark[1] \\
            & As a CTO I have designed, implemented and maintained the
              digital foundation for the company's continued growth.
              I have supervised and led development teams in Kenya, Canada, USA and Wales.\\
2014 - 2016 & \textbf{Software engineer, CERN}\footnotemark[1] \\
            & At CERN I was driving the development and maintenance of
              a monitoring toolchain for the accelerator control system complex.
              My contract was extended to work on a testbed for the Large Hadron
              Collider (LHC) accelerator, with an emphasis on real-time critical
              systems architecture in Java. \\
2013 - 2013 & \textbf{Instructor in the course \textit{mobile and distributed systemes}, IT-University of Copenhagen} \\
            & I lectured on principles and research on mobile and distributed
              systems and participated in the development of new teaching
              materials. \\
2012 - 2014 & \textbf{Student programmer, DemTech research group (\texttt{demtech.dk})}\footnotemark[1] \\
            & As a student programmer I developed a 3G based router system to
              measure the length of election queues. Used for nation-wide
              research and deployed at several polling places in 2013.
\end{tabularx}

\footnotetext[1]{A letter of recommendation can be sent per request.}

\section*{Volunteer work}
\begin{tabularx}{\textwidth}{l X}
2018 -      & Lead developer and maintainer of Volr, a Haskell-based environment
              for the construction, evaluation and analysis of neural networks
              (\texttt{https://github.com/volr/}). \\
2014 - 2016 & Tour guide for visitors around the CERN complex \\
            & As a guide I was introduced to the different parts of the
              control system in the accelerator complex, the cryogenics
              facilities and particle physics. I held presentations
              on all of the above for visitors from all around the globe. \\
2011 - 2014 & Active in the political student union at the IT University of Copenhagen. \\
2008 - 2015 & Co-creator and lead programmer of the collaborative drawing-tool
              RepoCad, now published as open-source (\texttt{https://github.com/selftiesoftware}).
\end{tabularx}

\section*{Papers}
\textit{Introducing RepoCad  - A prototype of the Internet of Digital Design}, eCAADe 33, Vienna 2015. \\
\textit{Predictable firm real-time Java}. Bachelor thesis, CERN and IT-University of Copenhagen, 2015. \\
\textit{Proxy voting in the European Union}. Bachelor thesis, Aarhus University, 2011.

%\section*{Passions}
%Cognitive science and philosophy of the mind. Has a personal goal to augment
%his cognitive capacities to better encompass the challenges (and solutions) of tomorrow.
%\vskip4pt
%\noindent
%Volunteer work. Believes in a strong social responsibility for the immediate community and society as a whole.
%\vskip4pt
%\noindent
%Device hacking. Rutinely changes the firmware and software of IT around him
%to support daily routines.
%\vskip4pt
%\noindent
%Music. Have been playing the violin since age 6 and piano since age 12.
%\vskip4pt
%%\noindent
%Hiking. Primarily in the Alps, Norway and Sweden.
%\vskip4pt
%\noindent

\end{document}
